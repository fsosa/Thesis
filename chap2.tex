\chapter{Background}

This section details the background of the TaleBlazer project, including the separate components of TaleBlazer and how it works as a whole. The history of TaleBlazer and past location-based projects are also detailed. Finally, this chapter expands on the need for TaleBlazer Analytics and prior analytics solutions. 

\section{TaleBlazer}

TaleBlazer is an augmented reality location-based game platform developed at the MIT Scheller Teacher Education Program (STEP). TaleBlazer is a platform in the true sense in that it is composed of multiple techologies which all come together to form the core of the TaleBlazer experience. At its core, TaleBlazer lets users create their own games that take place in the real world, using a block-based scripting language. Users can then choose to publish their games to the world at large. Using an Android or iOS mobile device, players play TaleBlazer games by downloading the game and physically walking around in the real-world location of the game and interacting with virtual game ``agents'': user-scripted virtual entities that are placed by the game designer at specific GPS coordinates.

\subsection{Past Projects}

TaleBlazer is the most recent iteration and study into AR games performed at the STEP lab. Past projects, such as MITAR and StarLogo TNG, provided a basis on which TaleBlazer was built, namely the emphasis on augmented reality and the use of a block-based scripting language.

\subsubsection{MITAR}

MITAR (MIT Augmented Reality) was the immediate ancestor of TaleBlazer. Similar to TaleBlazer, MITAR sought to let users play location-based augmented reality games on earlier mobile platforms, such as Windows Mobile. MITAR also focused on the educational impact of augmented reality games on users. One game, called ``Environmental Detectives'' put users into the roles of investigators searching for the source of a toxic spill, taking measurements in order to determine the environmental impact. [SOURCE THIS]

\subsubsection{StarLogo TNG}

StarLogo TNG (The Next Generation) was a project in programmable simulation modeling which allowed users to explore the workings of complex decentralized systems, such as ant colonies and traffic jams. [SOURCE THIS] Similar to TaleBlazer, users could program the simulation using a block-based scripting language. 

\subsection{TaleBlazer Technology}

The TaleBlazer platform is comprised of four main parts:
	\begin{itemize}
 		\item the game editor, which lets users create their own games
 		\item the mobile app, on which TaleBlazer games are played
 		\item the repository server, which stores and serves games
 		\item and the multiplayer server, which enables multiplayer TaleBlazer games
 	\end{itemize}

This software allows the user to play the role of the designer and create his/her own








