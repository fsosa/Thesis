\chapter{Introduction}

The explosion of the mobile market has led to the proliferation of location-aware mobile devices and a wide range of mobile applications (``apps'') that provide location-based content. Augmented reality (AR) applications, for example, enhance the user's real-life environment with location-specific information. The prevalence and affordability of these mobile devices, such as smartphones and tablets, make them a natural choice as a tool to augment education and learning. TaleBlazer is an augmented reality location-based games platform that allows users to create their own games that take place in the real world and play them on their mobile devices. A goal of the TaleBlazer project is to determine the education impact of location-based games. TaleBlazer Analytics is an automated system that allows game designers and researchers to gather and analyze anonymous data about users' behaviors during TaleBlazer games. 

\section{Motivations for TaleBlazer Analytics}
The wide gamut of user-created TaleBlazer games requires a data collection system that is both flexible and useful. TaleBlazer Analytics was developed to allow game designers and researchers to get specific metrics about when and how their games are played. For game designers, these metrics allow them to create more engaging and effective game experiences. For researchers, these metrics provide crucial information such as how players progress through the game and the choices that they make. Past data collection, game designers and researchers also need a way to quickly analyze the vast amount of gameplay data that is generated. 

Existing analytics solutions fail to gather data with the level of granularity required by the TaleBlazer project and often do not provide an adequate level of user privacy. Furthermore, the unique nature of TaleBlazer games requires custom analysis of the generated data. As a result, it was necessary to develop TaleBlazer Analytics to meet our needs. 

Over the past year and a half, I have developed an automated data collection system which seamlessly integrates with the existing TaleBlazer app and server architecture, working closely at each step with the TaleBlazer team. TaleBlazer Analytics is comprised of a backend server, mobile client, and web application, jointly responsible for the collection, storage, and analysis of TaleBlazer gameplay data. 

\section{Chapter Summary}
This thesis describes the background, design, and development of the TaleBlazer Analytics system. Chapter 2 details TaleBlazer in-depth and expands on the need for TaleBlazer Analytics. Chapter 3 explains the design process and preliminary decisions required for the development of TaleBlazer Analytics. Chapters 4, 5, 6, 7 detail the server, client, and web applications components of the project. Chapter 8 discusses the process of user testing and feedback that we received. Chapter 9 proposes future work for the project and Chapter 10 provides parting thoughts. 




